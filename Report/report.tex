\documentclass{article}

\usepackage[UKenglish]{babel}
\usepackage{helvet}
\renewcommand{\familydefault}{\sfdefault}
\usepackage{fancyhdr}
\usepackage{graphicx}
\usepackage{wrapfig}
\usepackage{csvsimple}
\pagestyle{fancyplain}
\fancyhf{}
\renewcommand{\headrulewidth}{0pt}
\fancyhfoffset[R]{0cm}
\rhead{\fancyplain{}{mk626}}
\lhead{\fancyplain{}{MichaelKubiak}}
\cfoot{\thepage}
\usepackage[margin = 1.5cm]{geometry}
\usepackage{xcolor}
\usepackage{natbib}
\setcitestyle{round}
\setcounter{secnumdepth}{0}

\begin{document}

	\vspace*{\fill}
		\begin{center}
			\huge\textbf{Title}
			
			\large\textbf{Author:} Michael Kubiak, University of Leicester
	
			\textbf{Date:} \today
			
		
		\end{center}
	\vspace*{\fill}

\pagebreak

	\section*{Abstract}

\pagebreak

	\tableofcontents

\pagebreak

	\section*{Introduction}
		Protein annotation is a highly important stage of the analysis of organisms.  There are a number of annotation schemes that work for different sets of proteins, including Enzyme commission (EC) number, which classifies enzymes by the reactions that they catalyse.  An EC number has 4 sections, separated by dots, each of which specifies the enzyme function to a greater degree, going from general reaction type down to specific substrate.  
		
		Another way that proteins are classified is by homology, providing families that evolved from a common ancestor.  Pfam \citep{RefWorks:doc:5d6e641de4b0a51fb0eed90f} holds profile sequences for families of proteins,  
		
	\section{Methods}
			
			
				
							
				  
	\section{Results}
		
				
	\section{Conclusion}
		
	\section{Future Application}
			
\pagebreak

	\bibliography{export.bib}{}
	\bibliographystyle{myplainnat}

\end{document}
